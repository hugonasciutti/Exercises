\documentclass[12pt]{article}
\usepackage[utf8]{inputenc}
\usepackage[brazilian]{babel}
\usepackage[T1]{fontenc}
\usepackage{amsmath}
\usepackage{amsfonts}
\usepackage{amssymb}
\usepackage{graphicx}
\usepackage{lmodern}
\usepackage{multicol}
\usepackage[a4paper,top=3cm,bottom=2cm,left=2cm,right=2cm]{geometry}
\newcommand{\limit}[2]{\lim\limits_{x\rightarrow #1}\,#2}
\newcommand{\ts}[1]{\tiny #1 \normalsize }
\newcommand{\nl}{\hfill \newline}

\begin{document}

%\begin{titlepage}
%\title{Resumo Matemática I}
%\author{João Barbosa de Souza Neto}
%\date{ }
%\maketitle
%\end{titlepage}

\tableofcontents

\newpage

\part{Matemática}

\section{Propriedades}

\subsection{Produtos Notáveis}

\begin{multicols}{2}
\begin{itemize}
\item $(a+b)^2 = (a^2+2ab+b^2)$
\item $(a-b)^2 = (a^2-2ab+b^2)$
\item $(a+b)^3 = (a^3+3a^2b+3ab^2+b^3)$
\item $(a-b)^3 = (a^3-3a^2b+3ab^2-b^3)$
\end{itemize}
\end{multicols}

\subsection{Fatoração}

\begin{multicols}{2}
\begin{itemize}
\item $a^2-b^2 = (a+b)(a-b)$
\item $a^3-b^3 = (a-b)(a^2+ab+b^2)$
\item $a^4-b^4 = (a-b)(a^3+a^2b+ab^2+b^3)$
\item $a^2+b^2 = a^2+b^2$
\item $a^3+b^3 = (a+b)(a^2-ab+b^2)$
\item $a^4+b^4 = a^4+b^4$
\end{itemize}
\end{multicols}

\begin{multicols}{2}
\begin{itemize}
\item $ax^2 + bx + c = a(x-x$\ts{1}$)(x-x$\ts{2}$)$
\item $2x^2-3x+1 = 2(x-1)(x-\dfrac{1}{2})$
\end{itemize}
\end{multicols}

\section{Matemágica}

\begin{multicols}{2}
\begin{itemize}
\item $-a+b = b-a = -(a-b)$
\item $-a-b = -(a+b)$
\item $a^2 + b = a(a + \dfrac{b}{a})$
\item $\sqrt{a} = \dfrac{a}{\sqrt{a}} \Rightarrow \dfrac{\sqrt{a}}{x} = \dfrac{a}{x\sqrt{a}}$
\item $\dfrac{1}{\sqrt{a}} = \dfrac{\sqrt{a}}{a} \Rightarrow \dfrac{x}{\sqrt{a}} = \dfrac{x\sqrt{a}}{a}$
\end{itemize}
\end{multicols}

\subsection{Exemplos Importantes}

\begin{multicols}{2}
\begin{itemize}
\item $1-x = -(x-1)$
\item $-x-1 = -(x+1)$
\item $1 = 1^n = \sqrt[n]{1}$
\item $-1 = -1^3 = \sqrt[3]{-1}$
\item $x^2-1 = (x-1)(x+1)$
\item $x^3-8 = (x-2)(x^2+2x+4)$
\item $x^2+x = x(x+1)$
\item $x^2+3xy = x(x+3y)$
\end{itemize}
\end{multicols}

\subsection{Fórmula de Bhaskara}

\begin{itemize}
\item $\Delta = -4ac$
\item $x = \dfrac{-b \pm \sqrt{\Delta}}{2a}$
\end{itemize}

\newpage

\section{Módulo}

\subsection{Propriedades}

\begin{itemize}
\item $|x| \geq 0$
\item $|x| = \sqrt{x^2}$
\item $|x| \cdot |y| = |x\cdot y|$
\item $|x|^2 = |x^2| = x^2$
\item $|x| < a = -a < x < a$
\item $|x| < -a = a < x < -a$
\end{itemize}

\subsection{Definição}

Para todo $x \in \mathbb{R}$:

\begin{itemize}
\item $|x| = 
\left\{\begin{array}{lcl}
\quad x & \text{se} & x \geq 0\\
-x & \text{se} & x < 0
\end{array}\right.$
\end{itemize}

\subsection{Exemplos Importantes}

Módulo da diferença:

\begin{itemize}
\item $ |x-a| =
\left\{\begin{array}{lcl}
\quad x-a & \text{se} & x-a \geq 0\\
-(x-a) & \text{se} & x-a < 0
\end{array}\right.
\Rightarrow
|x-a| = 
\left\{\begin{array}{lcl}
x-a & \text{se} & x \geq a\\
a-x & \text{se} & x < a
\end{array}\right.$
\end{itemize}
\nl
Módulo da soma:

\begin{itemize}
\item $ |x+a| =
\left\{\begin{array}{lcl}
\quad x+a & \text{se} & x+a \geq 0\\
-(x+a) & \text{se} & x+a < 0
\end{array}\right.
\Rightarrow
|x+a| = 
\left\{\begin{array}{lcl}
\quad x+a & \text{se} & x \geq -a\\
-x-a & \text{se} & x < -a
\end{array}\right.$
\end{itemize}

\subsection{Exemplos difíceis}

\begin{itemize}
\item $|2x-1| + |x+1| = $
\\[2cm]
\item $||x| - 1| = $
\end{itemize}

\newpage

\part{Funções}

\section{Funções}

\subsection{Domínio}

Para calcular o Domínio de uma função real $f(x)$, basta calcular quais números $x$ não pode assumir em $\mathbb{R}$, deste modo:

\begin{itemize}
\item $\dfrac{1}{a} \Rightarrow a \neq 0$
\item $\sqrt[n]{a} \Rightarrow a \geq 0$, sendo n par
\end{itemize}

\subsection{Propriedades}

\begin{itemize}
\item $f(x) = K$, o resultado é a constante $K$
\item $(f\pm g)(x) = f(x) \pm g(x)$
\item $(f\cdot g)(x) = f(x) \cdot g(x)$
\item $\dfrac{f}{g}(x) = \dfrac{f(x)}{g(x)} \text{, sendo } g(x) \neq 0$
\item $(f \circ g)(x) = f( g(x) )$ (Função Composta)
\end{itemize}

\subsection{Exemplo}

Na análise da função: $f(x)=\dfrac{|x-1|}{x-1}$
\\[0.2cm]
Devemos primeiro considerar o fato de que o domínio da mesma é $D=\mathbb{R}-\{1\}$.
\nl
Daí, segue que:

$$
f(x)=\dfrac{|x-1|}{x-1}\Rightarrow 
f(x)=
\left\{
\begin{array}{lcl}
\dfrac{x-1}{x-1} & \text{se} & x-1 > 0\\[0.5cm]
\dfrac{-(x-1)}{x-1} & \text{se} & x-1 < 0
\end{array}
\right.
\Rightarrow
f(x)=
\left\{
\begin{array}{lcl}
1 & \text{se} & x >  1\\[0.5cm]
-1 & \text{se} & x < 1
\end{array}
\right.
$$

\newpage

\part{Limites}

\section{Limites}

\subsection{Propriedades}

\begin{itemize}
\item $\limit{a}{mx+n} = ma+n $
\item $\limit{a}{x^n} = a^n$, sendo $n \in \mathbb{N}-\{0\}$
\end{itemize}

Sendo $\limit{a}{f(x)} = L$ e $\limit{a}{g(x)} = M$:

\begin{itemize}
\item $\limit{a}{C \cdot f(x)} = C \cdot L$
\item $\limit{a}{f(x)\pm g(x)} = L \pm M$
\item $\limit{a}{f(x) \cdot g(x)} = L \cdot M$
\item $\limit{a}{\dfrac{f(x)}{g(x)}} = \dfrac{\limit{a}{f(x)}}{\limit{a}{g(x)}} = \dfrac{L}{M}$, sendo $M \neq 0$
\item $\limit{a}{[f(x)]^n} = L^n$, sendo $n \in \mathbb{N}-\{0\}$
\end{itemize}

\subsection{Exemplos Importantes}

\begin{itemize}
\item $\limit{0}{\dfrac{1}{1+x}} = 1$
\item $\limit{a}{\dfrac{x^2-a^2}{x-a}} = \limit{a}{\dfrac{(x+a)(x-a)}{x-a}} = \limit{a}{x+a} = 2a$
\item $\limit{1}{x^3+3x^2+5x+7} = 1^3+3 \cdot 1^2+5 \cdot 1+7 = 1 + 3 + 5 + 7 = 16$
\end{itemize}



\end{document}